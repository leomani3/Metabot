\documentclass{report}
% Comment the following line to NOT allow the usage of umlauts
\usepackage[utf8]{inputenc}
\usepackage[francais]{babel}
\usepackage{listings}
\usepackage{xcolor}
\usepackage{textcomp}
\pagestyle{plain}
\definecolor{listinggray}{gray}{0.9}
\definecolor{lbcolor}{rgb}{0.9,0.9,0.9}
\lstset{
 backgroundcolor=\color{lbcolor},
 tabsize=4,
 rulecolor=,
 language=matlab,
 basicstyle=\scriptsize,
 upquote=true,
 aboveskip={1.5\baselineskip},
 columns=fixed,
 showstringspaces=false,
 extendedchars=true,
 breaklines=true,
 prebreak = \raisebox{0ex}[0ex][0ex]{\ensuremath{\hookleftarrow}},
 frame=single,
 showtabs=false,
 showspaces=false,
 showstringspaces=false,
 identifierstyle=\ttfamily,
 keywordstyle=\color[rgb]{0,0,1},
 commentstyle=\color[rgb]{0.133,0.545,0.133},
 stringstyle=\color[rgb]{0.627,0.126,0.941},
}

\title{Rapport de TER}
% Start the document
\begin{document}
\maketitle
\newpage
\tableofcontents
\newpage
% Create a new 1st level heading

\part{Présentation du projet}
\chapter{Introduction}
\section{But du projet}
\paragraph{}
L'objectif de ce projet est la réalisation d'un jeu basé sur un modèle multi-agent. L'idée générale du projet est dans la continuité du projet de l'année dernière sur le même thème. L'outil utilisé est Unity 3D, un moteur de jeu utilisé dans un grand nombre de réalisations de hautes qualités. Notre projet est utilisable sur Windows et Mac et pourrais être porté sur Android.
Ce projet consiste de réaliser un jeu que l'on peut qualifier de programmeur et de permettre, notamment, à de jeunes personnes de se familiariser avec le monde de la programmation. L'utilisateur pourra donc créer un comportement pour des robots appelé "unité" afin de remplir des objectifs du jeu.
\paragraph{}
Le projet Metabot est un projet modeste réalisé à partir du logiciel Unity 3D par un groupe d'étudiant débutant dans l'utilisation de cet outil. Malgré le peu d'expérience dans la création pure de ce genre de projet, le projet actuel est le fruit d'un travail important et d'une implication entière de toute l'équipe.
Le projet a donc pour unique prétention de communiquer notre amour du jeu vidéo et de la programmation.
\subsection{Cahier des charges}
\section{Notion d'agent}
\subsection{Système multi-agents}
\subsection{Representation dans notre projet}
\newpage
\chapter{WarBot: Le mode par défaut}
\section{Principe}
Dans WarBot, deux à quatre équipes se battent sur un terrain pour les ressources afin de survivre et d'éliminer les autres équipes afin d’être la dernière ne vie. Des ressources apparaissent sur la carte et peuvent etre converti en unité ou en soin.

\newpage
\part{Réalisation du projet}
\newpage

%%%
% PARTIE MOTEUR
%%%
\chapter{Partie "Moteur"}
\section{Phase de conception}
Pour réaliser ce projet, nous avons réalisé une conception basé sur notre savoir en matiere de programation orientée agent et tout en gardant à l'esprit la notion de généricité qui est au coeur de notre projet.
Le but de notre moteur de jeu est de permettre à des developpeur de pouvoir réaliser un mode de jeu orienté agent, de facon la plus simple possible.
\section{De l'étude de l'ancien projet à la refonte totale du moteur.}
\subsection{État de l'art de l'ancien projet.}


\subsection{Refonte du noyau.}

\section{Réalisation}
\subsection{Sur les unités.}

\subsubsection{Le corps}
\subsubsection{L'esprit}
Afin de pouvoir fonctionner, les unités possede 3 parties distinctes: La gestion des actions possibles, la gestion des perceptions et l'identité propre de l'unité.
\paragraph{Actions.}
Blablabla
\paragraph{Perceptions.}
Blablabla
\paragraph{Identité.}
Blablabla

\subsection{La liaison avec le comportement.}
\section{Fonctionnalités}
\section{Amélioration possible}

\newpage
%%%
% PARTIE Interface Graphique
%%%
\chapter{Partie Interface Graphique}
\section{Présentation}
La partie Interface Graphique comprend principalement l'habillage visuel des éléments avec lequel l'utilisateur va interagir pour jouer au jeu. MetaBot étant un jeu pour "programmeur", le joueur interagi surtout avec le menu principal et l'éditeur de comportement afin de créer des équipes pour pouvoir les faire s'affronter.
\paragraph{}
L'interface graphique que nous avons créé se décompose en quatre parties, le menu principal, le menu de paramètre, l'éditeur de comportement et des fonctionnalités directement en jeu.
\section{Etude de l'ancienne interface}
L'ancienne interface nous est tout de suite apparu comme plutôt vide et les actions y étaient très limitées, pour l'éditeur on pouvait seulement choisir une équipe, une unité et les Contrôle/Condition/Action basiques (Il y avait aussi plusieurs bugs présents mais on ne parle ici que de l'interface).
\paragraph{}
Le menu principal contenait un bouton de nouvelle partie, un bouton vers l'éditeur, un bouton pour quitter la partie et une case à cocher pour arrêter le son.
\paragraph{}
\paragraph{}
Le premier choix a donc été de revoir totalement l'interface, ce que nous pouvions faire grâce aux nouveaux éléments apportés. En effet il fallait rajouter de nouvelles fonctionnalités basiques à cette interface qui en avait finalement très peu et ajouter un peu de couleurs à tout ça.
\section{Nouvelle Interface}

\subsection{Menu Principal}
\subsubsection{Lancer une Partie}
Ce bouton permet de lancer une partie du jeu MetaBot. Le lancement de la partie prend en compte les équipes choisies, le nombre de chaque unités en début de partie, le nombre de ressources maximum (nombre de ressource présentes en jeu au même moment), et la carte de jeu.
\subsubsection{Bouton Editeur de Comportement}
Ce bouton permet d'accéder à l'éditeur de comportement.
\subsubsection{Bouton Paramètre}
Ce bouton ouvre (révèle) le menu des Paramètres.
\subsubsection{Choisir les équipes}
Dans le carré de chaque équipe il y a un menu déroulant avec les noms de toutes les équipes présentes dans les fichiers du jeu. On peut donc en sélectionner une pour qu'elle participe à la prochaine bataille.
De plus à coté du nom de l'équipe on retrouve aussi son score (ELO).
\subsubsection{Choisir le nombre d'équipe}
Au dessus de toute les équipes on peut choisir le nombre d'équipe participant à la partie. Les valeurs sont dans un menu déroulant et vont de 2 à 4.
\subsubsection{Bouton "Reload Team"}
Ce bouton permet de recharger les équipes.
\subsubsection{Bouton des Scores}
Le bouton en forme de couronne ouvre l'écran des scores. Chaque équipe a un score (ELO) qui indique son "niveau" et donc qu'elle est plus apte à gagner que les équipes avec un score plus faible.
\subsubsection{Bouton pour quitter le jeu}
Ce bouton ouvre une boite de dialogue demandant à l'utilisateur si il veut vraiment quitter le jeu. Il peut ainsi choisir de revenir sur le menu principal ou de fermer le jeu.
\subsubsection{Choisir carte de jeu}
On peut directement choisir le lieu de la partie en cliquant sur les flèches de part et d'autre de l’aperçu de la carte. Il existe pour le moment cinq cartes, Moutain, Plain, Simple, Desolate et Garden.
\subsubsection{Choisir nombre de départ de chaque unités}
En dessous de l’aperçu de la carte, il y a les noms des unités présente dans le jeu. Sous ces noms, le chiffre, indique le nombre de ce type d'unité présent au lancement de la partie. On peut incrémenter ou décrémenter ce chiffre à l'aide des boutons "+" et "-" à coté de celui-ci.
\subsubsection{Barre de Chargement}
Quand on lance une partie, une barre de chargement apparaît et indique l'avancement du chargement de la partie.

\subsection{Menu des Paramètres}
\subsubsection{Changer le Volume de la musique}
Ce slider indique le niveau sonore de la musique, on peut le changer en cliquant dessus et en déplaçant la valeur de 0 jusqu’à 100. 0 correspond à un arrêt de la musique et 100 au volume maximal. De plus le volume sonore dans le menu est le même dans l'éditeur de comportement et dans le jeu lui même.
\subsubsection{Activer/Désactiver le tir allié}
Ce bouton permet d'activer ou désactiver le fait que les unités d'une équipe peuvent infliger des dégâts aux autre unités de la même équipe en jeu.
\subsubsection{Choisir nombre de ressource maximum dans le jeu}
Dans cette case on peut entrer un chiffre entier qui indiquera le nombre maximum de ressource présente, en même temps, à l'écran en jeu.
\subsubsection{Choisir le mode de jeu}
Ce menu déroulant permet de choisir le mode de jeu de la prochaine partie. Il y a actuellement deux modes, le mode TestBot et le mode RessourceRace. Si le mode choisi est RessourceRace alors deux autres paramètres apparaissent, le temps avant que la partie ne s’arrête et la limite de ressource à atteindre pour gagner.
\subsubsection{Choisir la langue}
Les boutons en forme de drapeau indiquent les langues disponibles pour le jeu. Pour changer de langue il suffit d'appuyer sur le drapeau voulut et de faire Valider les paramètres.
\subsubsection{Bouton Retour}
Ce bouton permet de retourner à l'écran du menu principal.
\subsubsection{Bouton Valider}
Ce bouton valide les paramètres définit au dessus et revient au menu principal en ayant appliquer ces paramètres.

\subsection{Editeur de Comportement}
% ToolBox %
\subsubsection{Bouton Nouveau Comportement}
Ce bouton permet de vider le comportement de l'unité de l'équipe courante. Si on veut tout effacer sur l'unité où on est on appuie dessus au lieu de tout supprimer à la main.
\subsubsection{Bouton Chargement Comportement}
Ce bouton permet de charger un comportement pour l'unité de l'équipe courante.
\subsubsection{Bouton Sauvegarde du Comportement}
Ce bouton sauvegarde le comportement de l'unité de l'équipe courante. Si on change d'unité dans la même équipe mais sans sauvegarder, alors le comportement de l'unité précédente sera perdu.
\subsubsection{Bouton "Undo"}
Ce bouton permet de ramener la dernière pièce supprimée là où elle était. C'est le bouton d'annulation de changement.
\subsubsection{Bouton "Redo"}
Ce bouton permet d'annuler le dernier "Undo" en annulant son action et ramenant l'état du comportement.
\subsubsection{Bouton de retour au menu principal}
Ce bouton ramène l'utilisateur au menu principal. Avant de cliquer dessus il faut penser à bien sauvegarder le comportement en cours pour ne pas le perdre.

% Equipe + Unité %
\subsubsection{Choix de l'équipe}
Ce menu déroulant permet de choisir l'équipe sur laquelle on veut travailler.
\subsubsection{Choix de l'unité}
Ce menu déroulant permet de choisir l'unité de l'équipe sur laquelle on va opérer les changements de son comportement.
\subsubsection{Bouton Création d'équipe}
Juste à droite des équipes se trouve un bouton en forme de croix, il permet de créer un nouvelle équipe. Si on appuie une boite de dialogue s'ouvre et nous demande le nom de la nouvelle équipe. Après avoir validé le nom, la boite de dialogue se ferme et la nouvelle équipe est présente dans le menu déroulant des équipes.
\subsubsection{Bouton Suppression d'équipe}
Ce bouton permet de supprimer définitivement l'équipe courante. Une boite de dialogue demandera confirmation.
\subsubsection{Affichage des statistiques de l'unité courante}
Quand on choisit une unité, sur sa droite apparaît ses valeurs dans le cadre "Propriétés", cela correspond à ses statistiques.
\subsubsection{Affichage du modèle 3D de l'unité courante}
Dans le même cadre "Propriétés" apparaît aussi le modèle 3D de l'unité courante. C'est l'apparence qu'aura l'unité en jeu.

% Zone de création des comportements
\subsubsection{Boutons de choix de la catégorie de la pièce}
Sur la gauche, de manière verticale, se trouve cinq noms de catégories qui sont les Contrôles, les Conditions, les Actions, les Messages et les Actions non terminale. En cliquant sur une de ses catégories on affiche la liste des éléments de cette catégorie dans la zone directement à droite.
\subsubsection{Zone de sélection de la pièce suivant la catégorie}
C'est dans cette zone qu’apparaît la liste des éléments des catégories de pièces de l'éditeur. Les éléments sont présents sous forme de case avec leur nom à l'intérieur (certain possède même des menus déroulant pour choisir la valeur voulue une fois dans la zone d'édition). La couleur des éléments dépend de leur catégorie.
Les éléments peuvent être sélectionnés et déplacés dans la zone d'édition du comportement grâce au glissé/déposé (Drag & Drop).
\subsubsection{Zone éditeur de comportement de l'unité et de l'équipe courante}
Dans cette zone arrivent les pièces venant de la zone de sélection. Elles ce placent à l'aide du curseur de la souris sur un grille invisible. Si un pièce est valide alors elle est colorée sinon elle est grise. Les pièces "IF" possèdent un cadenas dans le coin supérieur droit, il permet de déplacer, en plus de la pièce IF, tout les éléments valides qui lui sont rattachés (hors IF). Les pièces de contrôle sont valide en dessous d'autre pièce du même type ou en dessous de la pièce "Start". Les autres pièces doivent se rattacher à des pièces de contrôle, en haut à droite pour les pièces Condition et en bas à droite pour les autres. Les pièces hors Contrôle sont ainsi valide lorsqu'elles sont au bon endroit, sur leur ligne collé au contrôle ou collé à une autre pièce valide. 

\subsection{Element dans le jeu}
\subsubsection{Réglage du volume du son}
En bas à droite de la fenêtre de jeu il y a un icône de son et un slider. Le slider permet, comme dans le menu des paramètres, de changer le volume du son. L’icône lui sert de bouton, si on le presse le son passe à 0 et l’icône devient barré. Si on appuie de nouveau il devient normal et le son revient comme avant.

\newpage
\chapter{Partie "Interpreteur"}
repasser sur le texte pour ajouter exemples et illustrations ! 
\section{Présentation et Attente}
La partie "Interpréteur" est la partie la moins visible du projet MetaBot,
mais il s'agit de la partie du projet servant de clé de voute du jeu.
Comme dit plus tôt, la particularité de MetaBot est que le joueur, qui pourra être considéré comme le programmeur, va préparer en amont une cohésion d'équipe à travers le comportement et va pouvoir lancer un match contre une autre équipe, afin d'évaluer quel comportement sera le meilleur.
L'interpreteur permet de faire la liaison entre l'éditeur du comportement ou l'utilisateur va développer son comportement, en utilisant un ensemble de d'instructions que nous avons prédéfinies  et la partie moteur, ou le fonctionnement des unités est inscrit, ainsi que les différents modes de jeux.
\paragraph{}
Le langage et l'ensemble des instructions nécessaires est alimenté par l'équipe Game Design qui nous a donné des exemples de messages ou de spécificités du langage qui pourrait être nécessaire. On pouvait ensuite tous en discuter en pesant le pour et le contre, afin de définir si la fonctionnalité allait être mise en place , et de quelle façon.
\section{Etude de l'ancien projet et récupération de ce qui est utile}
Au départ, il a été nécessaire de remettre en place un outil permettant de récupérer un comportement, qui était uniquement graphique, dans l'éditeur afin de pouvoir le renvoyer à la partie Moteur, pour que l'ensemble des unités puissent l'exécuter.
Nous avons pris connaissance de ce que l'ancien groupe avait mis en place et avons trié ce qui nous semblait correspondre à notre version du projet.
Il était obligatoire d'écrire le comportement récupéré de l'éditeur dans un fichier, afin de le récupérer , pouvoir le modifier , le déplacer , et le conserver pour plusieurs parties.
La solution mise en place par l'ancien groupe pour le stockage, qui était d'utiliser un fichier XML correspondait parfaitement à notre besoin, car la syntaxe et l'organisation en nœuds de ce genre de fichiers, permettait une récupération simple et claire des instructions. Nous avons ainsi pu récupérer une bonne partie de leur système d'écriture et de lecture de leur projet, tout en adaptant l'autre partie à nos besoins.
\paragraph{}
\newpage
La partie organisant les instructions à été complètement supprimée, et nous avons repensé un système plus générique et plus compréhensible pour les groupes qui vont récupérer ce projet plus tard.
\section{Conception et implémentation}
La construction des Instructions des unités dans la partie Interpréteur a grandement été facilitée et la liaison des instructions avec leur exécution a été décalée dans le moteur. Ainsi on ne manipule que des chaînes de caractères lors de l'accès dans le fichier, et le moteur effectue la liaison avec une structure de "Delegate" comme expliqué plus haut, rajoutant beaucoup de généricité au projet.
D'un point de vue de la sécurité, celle-ci a été décalée dans l'éditeur du comportement, afin de ne pas traiter des comportements erronés , mais prévenir a l'avance et même empêcher les erreurs d'apparaitre.
\paragraph{}
Le but de l'interpréteur était de traiter des fichiers de ce type :
\begin{lstlisting}[frame=single]
<behavior>
  <teamName>Default Team</teamName>
  <unit name="Explorer">
    <instruction>
    </instruction>
  </unit>
  <unit name="Base">
  </unit>
  <unit name="Light">
  </unit>
  <unit name="Heavy">
  </unit>
</behavior>
\end{lstlisting}
Ainsi nous avons 1 fichier par équipe écrite par l'utilisateur, chacun détaillant le comportement de chaque unité sous la même forme :
\begin{lstlisting}[frame=single]
<behavior>
  <teamName>Default Team</teamName>
  <unit name="Explorer">
 <instruction>
      <parameters>
        <PERCEPT_ENEMY />
      </parameters>
      <message>
        <ACTN_MESSAGE_HELP>
          <Light />
        </ACTN_MESSAGE_HELP>
      </message>
      <actions>
        <ACTION_MOVE />
      </actions>
    </instruction>
    <instruction>
      <parameters>
        <PERCEPT_BLOCKED />
      </parameters>
      <actions>
        <ACTION_MOVE_UNTIL_UNBLOCKED />
      </actions>
    </instruction>
  </unit>
</behavior>
\end{lstlisting}
Le comportement d'une unité est une suite d'instructions, dont l'ordre est très important. La hiérarchie va déterminer la priorité qu'aura l'action sur le tick. Ainsi l'action la plus prioritaire dont les conditions sont acceptées sera l'action effectuée sur ce tick.
Chaque instruction est toujours organisée de la même façon :
- Une liste de conditions à remplir pour que l'instruction soit considéré comme acceptée
- Une liste d'actions qui ne terminent pas l'action, appelées Actions non terminales, mais qui peuvent être les messages partagés entre les unités par exemple. Ces actions non terminales seront exécutés avant :
- L'action terminale, qui va être exécutée à la fin du tick pour l'ensemble des unités et va être l'action qui va être "physiquement" fait par l'unité, par exemple : Avancer, Tirer, Récupérer ou donner une ressource,Se soigner , ...
\paragraph{}
Comme dit précédemment, l'interpréteur relie les 2 parties majeures du projet.
Ainsi, une fonction permet de transformer un fichier ".wbt" (format des équipes WarBot) vers un comportement d'équipe utilisable par l'équipe moteur, et une fonction permettant la transformation d'un comportement récupéré de l'éditeur graphique vers un fichier XML, pour sa sauvegarde.
\newpage
\paragraph{}
Un nouveau travail d'interprétation a vite été ressenti dans le projet. En effet, le jeu étant destiné à aider les jeunes à appréhender la programmation, une traduction française était nécessaire.
Toujours dans l'idée de rester générique, un traducteur a été mis en place et attaché à l'ensemble des textes qui seront affichés à l'utilisateur.
\paragraph{}
//pas clair a changer
La traduction fonctionne ainsi :
Un premier script s'occupe de vérifier la langue qui doit être utilisée dans le fichier de configuration du jeu et va charger en mémoire un fichier  de traductions : 

Le fichier va être parcouru, et un ensemble de couples clé,valeurs vont être récupérés , la clé correspondant à la valeur initiale d'un bouton/ texte affiché, la valeur étant la traduction qu'il faudra affiché à la place de la clé.

Ce script est ainsi rattaché au "GameManager" , un objet Unity qui n'est pas détruit lorsque l'on change de scène. Ainsi les traductions sont conservées lorsqu'on passe de l'éditeur au menu principal , du menu principal aux cartes de jeu, ...
\paragraph{}
Un second script est lui attaché à tous les Objets du jeu qui vont devoir être traduits. Celui-ci va , à la création de l'objet auquel il est rattaché, chercher la langue actuellement affichée, récupérer comme clé le texte original de l'objet auquel le script est affilié, et va stocker la traduction, et l'afficher à l utilisateur.

En Runtime, si la langue est changée dans les paramètres du jeu, les script de traduction des objets vont mettre à jour leur langue, et récupérer la nouvelle traduction dans le "GameManager".
\newpage
\chapter{Partie "Game Design"}

\newpage
\part{L'avenir du projet}
\chapter{Amélioration possible}

% Uncomment the following two lines if you want to have a bibliography
%\bibliographystyle{alpha}
%\bibliography{document}

\end{document}
