\documentclass{report}
% Comment the following line to NOT allow the usage of umlauts
\usepackage[utf8]{inputenc}
\usepackage[francais]{babel}
\pagestyle{plain}


\title{Rapport de TER}
% Start the document
\begin{document}
\maketitle
\newpage
\tableofcontents
\newpage
% Create a new 1st level heading

\part{Présentation du projet}
\chapter{Introduction}
\paragraph{}
L'objectif de ce projet est la réalisation d'un jeu basé sur un modèle multi-agent. L'idée générale du projet est dans la continuité du projet de l'année dernière sur le thème. L'outil utilisé est Unity 3D, un moteur de jeu utilisé dans un grand nombre de réalisation de hautes qualités. Notre projet est utilisable sur Windows et Mac et pourrais être porté sur Android.
Ce projet consiste de réaliser un jeu que l'on peut qualifier de programmeur et de permettre, notamment, à de jeunes personnes de se familiariser avec le monde de la programmation. L'utilisateur pourra donc créer un comportement pour des robots appelé "unité" afin de remplir des objectifs du jeu.
\paragraph{}
Le projet Metabot est un projet modeste réalisé à partir du logiciel Unity 3D par un groupe d'étudiant débutant dans l'utilisation de cet outil. Malgré le peu d'expérience dans la création pure de ce genre de projet, le projet actuel est le fruit d'un travail important et d'une implication entière de toute l'équipe.
Le projet a donc pour unique prétention de communiquer notre amour du jeu vidéo et de la programmation.
\newpage
\chapter{MetaBot: Le mode par défaut}
\section{Principe}
Dans MetaBot, deux à quatre équipes se battent sur un terrain pour les ressources afin de survivre et d'éliminer les autres équipes afin d'etre la derniere ne vie. Des ressources apparaissent sur la carte et peuvent etre converti en unité ou en soin.

\newpage
\part{Réalisation du projet}
\newpage
\chapter{Partie "Moteur"}
\section{Phase de conception}
Pour réaliser ce projet, nous avons réalisé une conception basé sur les 
\section{De l'étude de l'ancien projet à la refonte totale du moteur}
\section{Réalisation}
\section{Fonctionnalités}
\section{Amélioration possible}

\newpage
\chapter{Partie "Graphisme"}

\newpage
\chapter{Partie "Interpreteur"}
repasser sur le texte pour ajouter exemples et illustrations ! 
\section{Présentation et Attente}
La partie "Interpréteur" est la partie la moins visible du projet MetaBot,
mais il s'agit de la partie du projet servant de clé de voute du jeu.
Comme dit plus tôt, la particularité de MetaBot est que le joueur, qui pourra être considéré comme le programmeur, va préparer en amont une cohésion d'équipe à travers le comportement et va pouvoir lancer un match contre une autre équipe, afin d'évaluer quel comportement sera le meilleur.
L'interpreteur permet de faire la liaison entre l'éditeur du comportement ou l'utilisateur va développer son comportement, en utilisant un ensemble de d'instructions que nous avons prédéfinies  et la partie moteur, ou le fonctionnement des unités est inscrit, ainsi que les différents modes de jeux.
\paragraph{}
Le langage et l'ensemble des instructions nécessaires est alimenté par l'équipe Game Design qui nous a donné des exemples de messages ou de spécificités du langage qui pourrait être nécessaire. On pouvait ensuite tous en discuter en pesant le pour et le contre, afin de définir si la fonctionnalité allait être mise en place , et de quelle façon.
\section{Etude de l'ancien projet et récupération de ce qui est utile}
Au départ, il a été nécessaire de remettre en place un outil permettant de récupérer un comportement, qui était uniquement graphique, dans l'éditeur afin de pouvoir le renvoyer à la partie Moteur, pour que l'ensemble des unités puissent l'exécuter.
Nous avons pris connaissance de ce que l'ancien groupe avait mis en place et avons trié ce qui nous semblait correspondre à notre version du projet.
Il était obligatoire d'écrire le comportement récupéré de l'éditeur dans un fichier, afin de le récupérer , pouvoir le modifier , le déplacer , et le conserver pour plusieurs parties.
La solution mise en place par l'ancien groupe pour le stockage, qui était d'utiliser un fichier XML correspondait parfaitement à notre besoin, car la syntaxe et l'organisation en noeuds de ce genre de fichiers, permettait une récupération simple et claire des instructions. Nous avons ainsi pu récupérer une bonne partie de leur système d'écriture et de lecture de leur projet, tout en adaptant l'autre partie à nos besoins.
\paragraph{}
La partie organisant les instructions à été complètement supprimée, et nous avons repensé un système plus générique et plus compréhensible pour les groupes qui vont récupérer ce projet plus tard.
\newpage
\chapter{Partie "Game Design"}

\newpage
\part{L'avenir du projet}
\chapter{Amélioration possible}

% Uncomment the following two lines if you want to have a bibliography
%\bibliographystyle{alpha}
%\bibliography{document}

\end{document}
